%!TEX root = ./main.tex
\begin{abstract}
    \abstractname
    \blankline
    %\bibliography{references}
    Electron microscopy was was first developed by Ernst Ruska in year 1956.
    Ever since its conception, electron microscopy has provided valuable insights into our world.
    It has now become one of the most powerful and versatile tool at scientists' disposal. 
    However as we know, electron microscopy uses electromagnetic lenses, which are highly aberrated.
    This results in image degradation and causes loss of lots of valuable information. 
    This problem has been solved almost completely by aberration corrected electron microscopes, which are however still considerably costly.

    Another problem routinely encountered by microscopes originate from the physics of image formation.
    Whenever an image is formed any information related to `phase' of scattering sample is lost.
    This information of phase is important for several reasons, most importantly it results in higher resolution as well as gives us information about sample thickness and elemental composition.

    Out of many methods suggested to recover this phase information and to remove effects of aberrations, one of the simplest is exit wave reconstruction. 
    It is simple to implement and highly effective.
    In following pages we would discuss fundamentals of imaging physics and exit wave reconstruction using Inverse Weiner Filter method, followed by detailed discussion of a simple exit wave reconstruction code written in MATLAB\textsuperscript{\textregistered}.

    Exit wave of electron microscope consists of symmetric and anti-symmetric part.
    We shall limit the discussion and implementation to symmetric part of exit wave only.
    The Inverse Weiner Filter method, among other concepts used here were already published in literature.\cite{meyer_symmetric,meyer_unsymmetric}
    \end{abstract}